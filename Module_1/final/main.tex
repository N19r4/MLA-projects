\documentclass[12pt,a4paper]{article}
\usepackage[T1]{fontenc}
\usepackage[polish]{babel}
\usepackage[utf8]{inputenc}
\usepackage{lmodern}
\selectlanguage{polish}
\usepackage{graphicx}
\usepackage{biblatex}
\usepackage{csquotes}
\addbibresource{bib.bib}
\begin{document}
\nocite{*}
\pagenumbering{gobble}
\clearpage
\begin{figure}[h]
\centering
\includegraphics{media/ps-logo.png}
\end{figure}
\hspace{3cm}
\begin{center}Sprawozdanie z modułu nr 1\end{center}
\begin{center}MLA 2025/2026\end{center}
\hspace{3cm}
\begin{center}\large\textbf{Aplikacje uczenia maszynowego \\w systemach interakcji człowiek-maszyna}\end{center}
\hspace{7cm}
\begin{flushright}Kierunek: Informatyka
\end{flushright}
\begin{flushright}Członkowie zespołu:
\par
\textit{Kinga Grabarczyk}
\par
\ldots
\par
\textit{Imię nazwisko}
\end{flushright}
\vfill
\begin{center}Gliwice, 2025/2026\end{center}\newpage
\pagenumbering{arabic}
\tableofcontents
\newpage
\section{Wprowadzenie}
\subsection{Zakres tematyczny sprawozdania}
\ldots 
\subsection{Zespół projektowy}
Imię i nazwisko :: Rola w projekcie :: Zadania projektowe
\newpage
\section{ML - zagadnienia teoretyczne}
\subsection{Wprowadzenie do części teoretycznej}
\ldots 
\subsection{Rozwinięcie}
\ldots 
\subsection{Podsumowanie części teoretycznej}
\newpage
\section{ML - zagadnienia praktyczne}
\subsection{Wprowadzenie do części praktycznej}
\ldots 
\subsection{Rozwinięcie}
\ldots 
\subsection{Podsumowanie części praktycznej}
\newpage
\section{Podsumowanie i wnioski}
\begin{itemize}
\item \textit{Podsumowanie}
\item \textit{Wnioski}
\end{itemize}
\newpage
\section{Spis literatury}
\printbibliography[heading=none] 
\end{document}
